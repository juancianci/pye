\documentclass[11pt]{article}

\usepackage[margin=1in]{geometry}
\usepackage{amsmath, amssymb, amsthm, mathtools}
\usepackage{bbm}
\usepackage{booktabs}
\usepackage{graphicx}
\usepackage{hyperref}

\title{A Market Formation Model for PYE}
\author{Kosmos Ventures}
\date{\today}

\begin{document}
\maketitle

\section{Overview}

We present a discrete-time economic model of a staking marketplace in which validators issue time-locked staking contracts that decompose deposited principal into a \emph{Principal Token (PT)} and a \emph{Yield Token (YT)}. The model is designed to study adoption, liquidity formation, pricing, and profit distribution over a multi-year horizon.

Time is indexed at the epoch level for reward generation and at the monthly level for reporting and aggregation. The model supports endogenous agent behavior, stochastic reward environments, secondary market trading, and configurable fee-allocation policies.

\section{Time and Horizon}

Let epochs be indexed by $t \in \mathbb{N}$, and months by $m \in \{1,\dots,24\}$.
Each month contains a fixed number of epochs $E$.

Define:
\[
\mathcal{T}_m := \{t : t \text{ belongs to month } m\}
\]

\section{Agents}

\subsection{Validators}

Let $\mathcal{V}$ denote the set of validators.
Each validator $v \in \mathcal{V}$ issues one or more staking contracts (``Pye Accounts'') indexed by $a \in \mathcal{A}_v$.

Each account $a$ is characterized by:
\begin{itemize}
  \item Principal $P_a > 0$
  \item Maturity epoch $T_a$
  \item Commission vector
  \[
  \mathbf{c}_a = (c_a^{\text{inf}}, c_a^{\text{mev}}, c_a^{\text{fee}}), \quad c_a^i \in [0,1]
  \]
\end{itemize}

\subsection{Stakers}

Let $\mathcal{S}$ denote the set of stakers.
Each staker $i \in \mathcal{S}$ is characterized by:
\begin{itemize}
  \item Risk aversion $\lambda_i \ge 0$
  \item Liquidity preference $\kappa_i \ge 0$
  \item Belief model for future reward streams
\end{itemize}

Stakers allocate capital across accounts and may trade PT and YT on secondary markets.

\subsection{Market Makers (Optional)}

Market makers provide liquidity in PT and YT markets and collect trading fees. Their behavior is abstracted unless microstructure is explicitly modeled.

\section{Reward Environment}

At each epoch $t$, the protocol generates per-unit-stake rewards from three sources:
\[
\mathbf{r}_t = (r_t^{\text{inf}}, r_t^{\text{mev}}, r_t^{\text{fee}})
\]

These processes are stochastic and may follow regime-switching dynamics. No parametric assumption is imposed beyond integrability.

\section{Yield Accrual}

For account $a$, the gross staking yield accrued during epoch $t < T_a$ is:
\[
Y_{a,t}^{\text{gross}} = P_a \left( r_t^{\text{inf}} + r_t^{\text{mev}} + r_t^{\text{fee}} \right)
\]

Validator revenue from commissions:
\[
Y_{a,t}^{\text{val}} = P_a \left(
c_a^{\text{inf}} r_t^{\text{inf}} +
c_a^{\text{mev}} r_t^{\text{mev}} +
c_a^{\text{fee}} r_t^{\text{fee}}
\right)
\]

Staker yield:
\[
Y_{a,t}^{\text{stk}} = Y_{a,t}^{\text{gross}} - Y_{a,t}^{\text{val}}
\]

\section{Tokenization}

Upon deposit of $P_a$:
\begin{itemize}
  \item $P_a$ units of PT are minted
  \item $P_a$ units of YT are minted
\end{itemize}

PT entitles the holder to $1$ unit of principal at maturity $T_a$.
YT entitles the holder to all staking yield generated by the principal until $T_a$.

\section{Secondary Markets}

Let $p_t^{PT}$ and $p_t^{YT}$ denote market prices.
Let $q_t^{PT}$ and $q_t^{YT}$ denote traded quantities during epoch $t$.

\subsection{Trading Volume}

Monthly notional trading volume:
\[
\text{Volume}_m =
\sum_{t \in \mathcal{T}_m}
\left(
p_t^{PT} q_t^{PT} + p_t^{YT} q_t^{YT}
\right)
\]

\subsection{Trading Fees}

Let $f \in (0,1)$ be the ad-valorem trading fee rate.
Monthly fees:
\[
\text{Fees}_m = f \cdot \text{Volume}_m
\]

\section{Fee Allocation Policy}

Trading fees are allocated according to a policy vector:
\[
\boldsymbol{\pi} = (\pi_V, \pi_S, \pi_P), \quad
\pi_V + \pi_S + \pi_P = 1
\]

Monthly fee revenues:
\[
\text{ValFee}_m = \pi_V \cdot \text{Fees}_m
\]
\[
\text{StkFee}_m = \pi_S \cdot \text{Fees}_m
\]
\[
\text{ProtFee}_m = \pi_P \cdot \text{Fees}_m
\]

\section{Profit Accounting}

Define monthly totals:
\[
\text{GrossYield}_m = \sum_{t \in \mathcal{T}_m} \sum_a Y_{a,t}^{\text{gross}}
\]
\[
\text{ValYield}_m = \sum_{t \in \mathcal{T}_m} \sum_a Y_{a,t}^{\text{val}}
\]
\[
\text{StkYield}_m = \text{GrossYield}_m - \text{ValYield}_m
\]

Monthly profits:
\[
\Pi_m^{V} = \text{ValYield}_m + \text{ValFee}_m - C_m^V
\]
\[
\Pi_m^{S} = \text{StkYield}_m + \text{StkFee}_m - C_m^S
\]
\[
\Pi_m^{P} = \text{ProtFee}_m - C_m^P
\]

Total profit:
\[
\Pi_m^{\text{Total}} =
\Pi_m^{V} + \Pi_m^{S} + \Pi_m^{P}
\]

\section{Velocity}

Define deposits $D$ as average outstanding principal.

Define monthly velocity:
\[
V := \frac{\text{Volume}_m}{D}
\]

\section{Deposits--Velocity Profit Matrix}

For scenario inputs $(D,V)$, cumulative 24-month trading volume:
\[
\text{Volume}_{24} \approx 24 \cdot V \cdot D
\]

Cumulative fees:
\[
\text{Fees}_{24} = f \cdot 24 V D
\]

Let $y_{24}$ denote cumulative staking yield factor over 24 months.

Total profit:
\[
\Pi_{24}^{\text{Total}}(D,V) =
D y_{24} + f \cdot 24 V D - C_{24}
\]

Assuming average validator yield share $\alpha \in [0,1]$:
\[
\Pi_{24}^{V} = \alpha D y_{24} + \pi_V f \cdot 24 V D
\]
\[
\Pi_{24}^{S} = (1-\alpha) D y_{24} + \pi_S f \cdot 24 V D
\]
\[
\Pi_{24}^{P} = \pi_P f \cdot 24 V D
\]

\section{Reported Outputs}

The simulator reports:
\begin{itemize}
  \item Monthly trading volume $\{\text{Volume}_m\}_{m=1}^{24}$
  \item Monthly trading fees $\{\text{Fees}_m\}_{m=1}^{24}$
  \item Monthly profit distribution $(\Pi_m^{V}, \Pi_m^{S}, \Pi_m^{P})$
  \item Deposits $\times$ Velocity profit matrix (total and split)
\end{itemize}

\section{Experimental Results}

\subsection{Base Case Parameters}

The following parameters define the base case simulation scenario:

\begin{table}[h]
\centering
\begin{tabular}{@{}ll@{}}
\toprule
\textbf{Parameter} & \textbf{Value} \\
\midrule
Simulation horizon & 24 months (720 epochs) \\
Epochs per month & 30 \\
Initial deposits & \$10,000,000 \\
Number of validators & 5 \\
Number of stakers & 20 \\
Trading fee rate $f$ & 0.30\% (30 bps) \\
Fee allocation $(\pi_V, \pi_S, \pi_P)$ & (0.40, 0.30, 0.30) \\
Base trading velocity & 15\% monthly \\
Validator cost (monthly) & \$1,000 \\
Protocol cost (monthly) & \$5,000 \\
\midrule
Base inflation rate & 4.5\% annual \\
Base MEV rate & 2.0\% annual \\
Base protocol fee rate & 1.0\% annual \\
\bottomrule
\end{tabular}
\caption{Base case simulation parameters}
\label{tab:params}
\end{table}

\subsection{Summary Statistics}

Over the 24-month simulation horizon, the model produces the following aggregate results:

\begin{table}[h]
\centering
\begin{tabular}{@{}lr@{}}
\toprule
\textbf{Metric} & \textbf{Value} \\
\midrule
Average deposits $\bar{D}$ & \$7,083,333 \\
Average monthly velocity $\bar{V}$ & 7.36\% \\
Cumulative gross yield & \$1,152,694 \\
Cumulative trading volume & \$13,129,104 \\
Cumulative trading fees & \$39,387 \\
\midrule
Cumulative validator profit $\sum \Pi^V$ & \$63,651 \\
Cumulative staker profit $\sum \Pi^S$ & \$1,092,614 \\
Cumulative protocol profit $\sum \Pi^P$ & $-$\$108,184 \\
\textbf{Cumulative total profit} $\sum \Pi^{\text{Total}}$ & \$1,048,081 \\
\midrule
Yield factor $y_{24}$ & 16.27\% \\
Average validator share $\alpha$ & 6.24\% \\
\bottomrule
\end{tabular}
\caption{Cumulative simulation results over 24 months}
\label{tab:summary}
\end{table}

\subsection{Monthly Profit Distribution}

Table~\ref{tab:monthly} presents monthly profit distribution for the first 12 months. Trading volume and fee revenue decline over time as accounts mature and deposits decrease.

\begin{table}[h]
\centering
\small
\begin{tabular}{@{}rrrrrr@{}}
\toprule
\textbf{Month} & \textbf{Volume} & \textbf{Fees} & $\Pi^V$ & $\Pi^S$ & $\Pi^P$ \\
\midrule
1  & \$759,212  & \$2,278 & \$3,509  & \$56,322  & $-$\$4,317 \\
2  & \$694,878  & \$2,085 & \$4,209  & \$65,950  & $-$\$4,375 \\
3  & \$697,352  & \$2,092 & \$3,724  & \$59,436  & $-$\$4,372 \\
4  & \$769,811  & \$2,309 & \$4,110  & \$62,881  & $-$\$4,307 \\
5  & \$774,505  & \$2,324 & \$3,916  & \$59,329  & $-$\$4,303 \\
6  & \$804,798  & \$2,414 & \$4,120  & \$63,259  & $-$\$4,276 \\
7  & \$766,434  & \$2,299 & \$3,772  & \$58,319  & $-$\$4,310 \\
8  & \$667,123  & \$2,001 & \$3,483  & \$57,531  & $-$\$4,400 \\
9  & \$789,954  & \$2,370 & \$3,897  & \$59,494  & $-$\$4,289 \\
10 & \$693,086  & \$2,079 & \$3,954  & \$62,323  & $-$\$4,376 \\
11 & \$711,411  & \$2,134 & \$3,787  & \$60,418  & $-$\$4,360 \\
12 & \$707,804  & \$2,123 & \$3,914  & \$61,331  & $-$\$4,363 \\
\bottomrule
\end{tabular}
\caption{Monthly trading volume, fees, and profit distribution (months 1--12)}
\label{tab:monthly}
\end{table}

\subsection{Deposits--Velocity Profit Matrix}

The profit matrix in Table~\ref{tab:profitmatrix} shows projected 24-month total profit $\Pi_{24}^{\text{Total}}$ for various combinations of average deposits $D$ and monthly velocity $V$.

\begin{table}[h]
\centering
\begin{tabular}{@{}r|rrrrrr@{}}
\toprule
\textbf{Deposits} & $V=5\%$ & $V=10\%$ & $V=20\%$ & $V=50\%$ & $V=100\%$ & $V=200\%$ \\
\midrule
\$1M   & \$22K   & \$26K   & \$33K   & \$55K   & \$91K    & \$163K \\
\$5M   & \$688K  & \$706K  & \$742K  & \$850K  & \$1.03M  & \$1.39M \\
\$10M  & \$1.52M & \$1.56M & \$1.63M & \$1.84M & \$2.20M  & \$2.92M \\
\$25M  & \$4.01M & \$4.10M & \$4.28M & \$4.82M & \$5.72M  & \$7.52M \\
\$50M  & \$8.17M & \$8.35M & \$8.71M & \$9.79M & \$11.59M & \$15.19M \\
\$100M & \$16.49M & \$16.85M & \$17.57M & \$19.73M & \$23.33M & \$30.53M \\
\bottomrule
\end{tabular}
\caption{Total profit matrix $\Pi_{24}^{\text{Total}}(D,V)$ over 24-month horizon}
\label{tab:profitmatrix}
\end{table}

\subsection{Key Observations}

\begin{enumerate}
  \item \textbf{Staker dominance:} Stakers capture the majority of total profit (104\% of total, offsetting protocol losses), driven primarily by yield accrual rather than fee revenue.

  \item \textbf{Protocol breakeven threshold:} At the base case parameters, the protocol operates at a loss ($-$\$108K over 24 months). Profitability requires either:
  \begin{itemize}
    \item Higher trading velocity ($V > 50\%$ monthly), or
    \item Larger deposit base ($D > \$50$M with current velocity), or
    \item Increased protocol fee share $\pi_P$
  \end{itemize}

  \item \textbf{Velocity sensitivity:} The profit matrix demonstrates that velocity has a multiplicative effect on fee-derived profits. Doubling velocity from 100\% to 200\% increases total profit by approximately 30\% at \$100M deposits.

  \item \textbf{Deposit decay effect:} Monthly profits decline as staking accounts mature (months 13--24 show reduced volume), highlighting the importance of continuous deposit inflows for sustained protocol revenue.

  \item \textbf{Validator economics:} Validators earn modest but consistent profits (\$63K cumulative) through commission income, with low sensitivity to trading activity.
\end{enumerate}

\end{document}
